\chapter{Data sources \label{chap:data_sources}}

\Chapref{chap:data_sources} provides a brief overview of the data sources used in this report and the ANZSIC code structure which is the basis of discrete market analysis throughout the report.

\section{Australian and New Zealand Standard Industrial Classification (ANZSIC)}
ANZSIC was developed as a means of grouping similar business activities for the primary purpose of data collection and analysis. 
Industries are classified into 19 divisions (1-digit), 86 sub-divisions (2-digit), 214 groups (3-digit), and 506 classes (4-digit).

It is common for firms to have business activities spanning multiple industries.
Some large corporations, such as Wesfarmers, operate in a number of different industries using subsidiary companies or different brand names. 
In other cases there are economies of scope to operating across multiple industries, which explains why electricity retailers tend to also be gas retailers, or why mobile network operators tend to also provide internet services.

Our analysis of industry concentration focuses on the most granular ANZSIC measure: the 4-digit industry class. 
This is because competition is usually most intense between similar business activities.

There are limitations to using ANZSIC classes. 
For instance, an industry may face a degree of external competition, such that a high level of concentration within a particular industry may not be associated with a lack of competitive pressure. 
Similarly, some industry definitions may be too broad to capture pockets of concentration or non-competitiveness for specific business activities.

