%!TEX root=../Report.tex
\chapter{Keeping the pressure on: what policy makers should do \label{chap:policy}}


%The main contribution of this report is to set out evidence about levels and changes in market power, but the frameworks used provide a useful basis to summarise policy directions and priorities. In all of these areas there is a great deal of complexity. The observations presented here are high-level...


The competition policy agenda for the three groups of sectors with barriers to entry should be to tighten regulation of natural monopolies; to lift the regulatory burden more broadly across the economy, but toughen pro-competition regulation; and to reduce barriers to entry in the scale-economy sectors. Government should also make it easier for consumers to compare and switch providers, and adapt policy to technology and disruption.
% \textbf{implement the Harper Review recommendations on institutions}.

% \begin{itemize}
% \item \textbf{Get Harper across the line};
% \item \textbf{Tighten regulation of natural monopolies};
% \item \textbf{Prune regulation -- or toughen it};
% \item \textbf{Reduce barriers to entry in the scale-economy sectors};
% \item \textbf{Help consumers navigate \& cut switching costs};
% \item \textbf{Adapt to technology \& disruption}.
% \end{itemize}

\section{Tighten regulation of natural monopolies}

Natural monopolies face little direct competition, by definition, so many of them are regulated with the aim of constraining prices while still providing strong commercial incentives for investment and innovation. Commonwealth agencies regulate some natural monopolies (for example, telecommunications); states regulate others (for example, most ports). 

But the needed regulation is not working well everywhere. The high profits earned in electricity distribution and transmission, in ports, in wired telecoms, and in some airports suggest that regulation may be too lax in these sectors. Practitioners have also highlighted deficiencies in the regulation of natural monopolies:

\begin{quote}
    `The preference for price monitoring of privatised monopolies is a big part of the problem. In the absence of competition, merely monitoring prices makes little to no difference. Price monitoring does not amount to regulation.'

    \rightline{-- Rod \textcite{Sims-2016-Privatisation-Regulation}}
\end{quote}     
%\begin{quote}
%    \textbf{`It is becoming increasingly apparent that the threat of regulation under the National Gas Law is not acting as an effective deterrent to monopoly pricing and other exercises of market power.'}
%\end{quote}     
%\vspace{-3pt}
%\rightline{-- Rod \textcite{Sims-2016-Keynote-Gas}}

Governments cannot increase competitive pressure on natural monopolies, but they can improve the performance of natural monopolies by:
\begin{itemize}
%\item \hl{Something overall: the response to the national access regime, and  about price monitoring}. 
\item Cutting prices in \textbf{electricity distribution and transmission}. Previous work has shown how poor regulation has failed to put pressure on returns to operators.%
\footnote{\textcites{WoodHunterOTooleEtAl2012}{AERdeterminations}. The most recent regulatory determinations are beginning to apply pressure.}
\item Toughening price and access regulation in \textbf{ports}. Some ports negotiate access and prices with a small number of commercial customers, and price negotiations that are not backed by the alternative of arbitration will reflect the often unbalanced bargaining power of customer and supplier. Other ports set prices to a broader set of users, subject to a regulated cap or to competitive pressure from nearby ports. In either case, high prices can result if regulation does not sufficiently compensate for weak competitive pressure.%
\footcites{ACCC-ports-reg-2016}{Vic-port-reg-2015}{QLD-ports-2017}{ESC-ports-2014}
\item Writing off enough of the National Broadband Network to ensure pricing permits efficient use of \textbf{fixed-line telecommunications}. The wholesale prices of many fixed-line voice and data services are directly regulated by the ACCC, as are retail prices for voice-line rental and some calls. A 2015 ACCC pricing decision reduced the regulated prices in fixed-line telecoms.%
\footnote{The regulated price was reduced in part because the ACCC no longer set its pricing to cover the costs of building a new network (\textcite{ACCC_fixedlineprice_2015}).}  
The NBN is becoming the main provider of fixed-line services. If its costs prove too high, prices will also be inefficiently high if they are set to maximise cost recovery.%
\footcite{ACCCCommsMarketStudyDraft2017}
\item Setting clear conditions for \textbf{airports}, under which regulators should move from price monitoring to price regulation. There is increasing concern that the ACCC-administered price monitoring regime is too weak. That regime has not done much to constrain market power in some capital city airports, though some have continued to invest strongly.%
\footcites{PC-airport-2012}{ACCC_Airports_2015-16}{ABC_sims-airport_2017}
\item No longer boosting prices of \textbf{privatised assets} by limiting competition or regulation at the expense of users.%
\footcite{ACCC_privatise_smart_2016}
\end{itemize}

\section {Lift the regulatory burden more broadly, but toughen pro-competition regulation}

% Regulation can limit competitive pressure directly by favouring some firms, or indirectly by imposing costs on all firms.%
% \footcites{Econsuperstar2016}{Promarket2017}
% While indiscriminate deregulation is not justified, as some regulation can intensify competition or serve other worthwhile goals, there is much to be done. 

In the heavily-regulated sectors, the government should: cut the overall regulatory burden; make it easier for banking customers to switch, and for financial services competitors to enter; intensify pricing pressure on health insurers; permit stronger competition in pharmacies; and set higher licence prices for sports betting and casinos. 

\begin{itemize}
\item Cut the \textbf{overall regulatory burden} by removing constraints on entry and exit, cutting preferential treatment of firms, and reducing impediments to efficient allocation of labour and capital. The main opportunities in these areas, identified by the Harper Review and the Productivity Commission, include reforming the industrial relations system, aligning Australian product standards with those in other major markets, relaxing restrictions on retail trading hours, and mandating company director identification numbers. Other opportunities include reviewing industry assistance, improving government procurement, relaxing restrictions on cabotage and shipping, improving trade in books and second-hand cars, implementing the National Water Initiative, and reviewing competition in the gas market.%
\footnote{\textcites{Harper2015Competition}{PC-shiftthedial-2017}.}
\item In \textbf{banking}, governments should make it easier for customers to switch banks, and make it easier for new competitors to enter the market. Government could cut switching costs by making it easier for customers to share their data, to transfer their direct debits to a new bank, and to free-up their data from the control of their current bank.%
%They can pave the way for new competitors by making sure that .%
\footnote{The Productivity Commission is conducting an inquiry into competition in the financial system (\textcite{PC-FScompetition-2017}) and Treasury is reviewing policy options for `open banking' (\textcite{Treasury-openbanking-2017}).}
Competition should also be strengthened in other parts of the financial services industry, including superannuation and foreign exchange.%
\footnote{\textcites{MinifieSavageCameron-2015-Super-savings}{Forexfees2017}.}
\item In \textbf{health insurance}, APRA and government should increase the pressure they apply on premiums, and should consider giving premium approvals less frequently.%
\footnote{See \textcite[][Chapter~9]{LaffontTiroleTheory1993} for a discussion of how repeatedly reset short-term prices can deter regulated firms from reducing costs.} 
\item In \textbf{pharmacies}, government should finally remove constraints on competition, as many reviews have urged.%
\footcite{PC-shiftthedial-2017}
\item In \textbf{sports betting and casinos}, governments concerned about problem gambling may prefer not to issue more licences, even though that would reduce the super-normal returns some incumbents earn. Instead, governments should review options to get better public value from existing licences, for example by auctioning them.%
\footnote{The Victorian Government auctioned gaming machine licences in 2008, and in future will take a share of revenue (\textcite{Vic-gaming-machine-2017}).}

\end{itemize}

\section{Reduce barriers to entry in the scale-economy sectors}
Policy makers can help intensify competition in the scale-economy sectors, but there are no silver bullets. About half of the profits earned there are super-profits, suggesting that competitive pressure is weak or takes effect only gradually (\Chapref{chap:profits}).
The case for further changes to competition laws, however, is not strong now.
Protecting competitive pressure in the scale-economy sectors is `core business' for the ACCC\@.
It applies the Competition and Consumer Act in seeking to protect competition by preventing the misuse of market power, preventing cartels and other concerted practices, and preventing mergers that would lead to a substantial lessening of competition. The Harper Review's agenda for changes to the competition laws has been largely delivered, though there may be scope to increase penalties for some breaches of the law.%
\footcites{Harper2015Competition}{Morrison2017Competition}{Caronpenalties2017}
%but nothing in competition law can \textbf{intensify} competitive pressure in markets where firms follow the law.

That leaves policies that may intensify competition across the economy (as discussed elsewhere in this chapter) or in individual sectors.%
%Plus develop policies that work for online platforms (see below).

\textbf{Supermarkets} remain highly concentrated and highly profitable. While the incumbents have built profitable businesses with large market shares in liquor and petrol retailing, they have lost market share to new entrants in their core supermarket businesses (\Chapref{chap:trends}). One option that might intensify competitive pressure is to relax zoning restrictions that can limit the entry of competitors.%
\footnote{\textcites{ACCC-grocery-2008}{PC-retail-2011}. Some submissions to the Harper Review proposed changing merger laws to constrain `creeping acquisitions' (where a firm gains market power through a series of small individual transactions that individually do not result in a substantial lessening of competition). The Harper Review considered and rejected the proposed changes (\textcite{Harper2015Competition}).}


\textbf{Mobile telecoms} is a concentrated and highly profitable sector, but the networks are investing strongly, and the price of data service is falling fast. The ACCC's draft communications market study found that there was adequate competition. Competitive intensity may not change much unless someone builds a fourth network, but there may be little government can do to encourage that. The ACCC recently rejected applications to subject mobile networks to an access regime. Policy makers should ensure that 5G networks are allowed to compete freely with the NBN.% 
\footcites{ACCCCommsMarketStudyDraft2017}{ACCC-mobile-services}
%\textbf{ISPs} Presumably subject to the communications market study. 

\section {Make it easier for consumers to compare and switch providers}

Many consumers find it complex and confusing to compare providers of retail energy, mobile telecoms, mortgages, and superannuation. Consumers can also find switching providers cumbersome and costly. As a result, many pay high prices in such `confusopolies'.%
\footnote{Retail energy: \textcites{WoodBlowers-2017-price-shock}{Bendavid-cosmonaut-15}; mobile telecoms: \textcite{Gans-2005-Confusopoly}; mortgages: \textcite{Bankhaggle_2017}; and superannuation: \textcite{MinifieSavageCameron-2015-Super-savings}.} 
 
Governments can improve market functioning by mandating that providers share information so customers can compare. Governments can reduce customer-switching costs through initiatives such as mobile number portability. And governments should design more wholesale forms of competition to increase competitive intensity in markets such as superannuation where members are highly disengaged.%
\footcite{MinifieSavageCameron-2015-Super-savings}

\section{Adapt policy to technology and disruption}

Three major technology shifts are changing competition and challenging policy makers. First, online platforms have developed in media, search and retail. The larger platforms have developed significant pricing and market power.% 
%A large user base can sustain revenue even at much higher prices than smaller competitors charge. It also permits them to improve their services and build new ones. They may defend their market position using strategies that other types of firms do not use as much (for example, buying a smaller platform to prevent its user base being acquired by a potential competitor).%
\footcites{EconData2017}

\CenturyFootnote

Second, data is becoming an important source of competitive advantage. Online platforms amass data on customers, and that can reinforce their competitive positions. More generally, control of data is now central to how businesses retain and derive revenue from their customers. \footcite{EconData2017}

Third, firms increasingly set their prices by machine. Algorithms can set prices for individual customers based on extensive knowledge of their likely income and preferences. They can also set prices based on what competitors charge, and so make it easier to settle on practices that limit price competition. 

Government can improve the quality of competition in such online, data-intense and automated markets by: 
\begin{itemize}
\item Mandating that customers can take their data with them to another provider.%
    \footcites{Gruen_portable_2014}{Zingales_portability_2017}{Harford_Facebook_2017}
\item Mandating that data not be withheld by firms with market power to the detriment of competition (for example, mandating that car manufacturers share data essential for servicing the cars).%
    \footcites{Newyorker_algo_2015}{EzrachiVirtual}{EconPricebots2017}
\item Monitoring algorithmic pricing for evidence of tacit collusion, and even requiring companies to share their computer code for forensic examination.%
    \footcites{ACCCNewCarMarketStudyDraft2017}
\item Giving weight to the value of data in mergers and acquisitions, even if the owner of the data is small by traditional measures such as market share or revenue.%
    \footcite{EconData2017}
\item Making available to consumers government services now available only to businesses.%
    \footcites{EconData2017}{Gruen_centralbanking_2014}
\end{itemize}

Governments around the world, including Australia's, are examining ideas like these, but few have implemented policies based on them.%
    \footcites{PC-data-2017}{FT_Japan_competition_2017}{OECD-Algorithm-2017}


% \section{Build supporting institutions}

% The Government has begun to implement the recommendations of the Harper Review.  It needs to work closely with the states, and to institutionalise the  . While much of the remainder of the Harper agenda is for the % 
% \footnote{The Senate passed changes to competition law in November 2017, including changes to the prohibition on the misuse of market power and on concerted practices, changes to ACCC approvals processes (for mergers, collective bargaining, and boycotts by small business), and changes to the National Access Regime  (\textcite{Morrison2017Competition}). The Government also commissioned two Productivity Commission inquiries. The first, into competition, contestability and choice in human services, is complete. The second, into competition in financial services, has just started.} It particular it should:

% \begin{itemize}

% \item Establish a body to oversee progress on competition policy. 
% \item Establish a new Competition Principles Agreement with the states, supported by incentive payments.
% % \item Remove restrictions on competition in pharmacies.
% % \item Increase domestic customers access to international air carriers flying domestically.%
% % \footnote{It may be wise to undertake such changes as part of trade deals.} 
% %\hl{permitting parallel imports including for used cars }.%
% \end{itemize}

\section{Summing up}

Contrary to widespread belief, the market power of firms in large, concentrated sectors in Australia is not higher than in most countries. Neither has it changed much in the past 15 years. Barriers to entry and market concentration account for only a small fraction of the variation in profit across firms. But firms in some sectors that are protected by barriers to entry do earn persistent high profits. They pass on higher prices to their consumers (or lower prices to their suppliers) that total about \$16 billion, or 1 per cent of GDP. 

Governments should seek to intensify competitive pressure in the private economy. While existing policy settings and laws have limited the accumulation and misuse of power by large firms across much of the economy, the patterns of concentration and super-profits behind barriers to entry suggest that governments and regulators can do better.

No single major policy change would strongly increase competitive intensity in Australia's non-traded economy. But there is much governments can and should do.
%; \textbf{implement the Harper Review recommendations on institutions}.


% \section {Counter rent-seeking} 

% Finally, governments should also take steps to limit rent-seeking. There is no single step. 

% But how? The Hayek approach is for govt to stick to principles, do fewer arbitrary changes. But that cat seems to be out of that bag

% Options
% a) limit revolving door
% b) more transparency on decisionmaking, political donations
% c) bigger penalties for corruption
% d) federal ICAC
% e) auditor / evaluator general on industry policy

% It is not always a bad thing that departing ministers, their advisers, and public servants then go on to work in the industries they regulate. But such revolving doors create strong incentives for public decisionmakers to act in the interests of their potential future employers. 

%   \item firms with market power may lobby governments to introduce or retain regulations that entrench barriers to entry (\hl{the economic cost of rent-seeking is potentially much higher than the static cost of mark-ups -- potentially as high as the super-normal profits}, \url{http://cies.org.pe/sites/default/files/cursos/files/posner_1975.pdf}) 
%

% The captured economy (\url{https://global.oup.com/academic/product/the-captured-economy-9780190627768?cc=au&lang=en&})

% Game of mates (see \url{https://www.amazon.com.au/Game-Mates-favours-bleed-nation-ebook/dp/B06Y1WF2BC} and \url{http://www.cis.org.au/app/uploads/2017/09/33-3-tunny-gene.pdf} for a critical review)

% Piece on rents in Aust

% \url{https://promarket.org/suspect-major-reason-rise-concentration-technological-change-particularly/}

% \url{http://www.nber.org/papers/w23356}

% \url{https://www.aeaweb.org/articles?id=10.1257/jep.31.3.113}

% \url{https://papers.ssrn.com/sol3/papers.cfm?abstract_id=2778641}
% (Substantiate for Australia?)


% https://www.claytonutz.com/knowledge/2017/october/major-shake-ups-to-australian-competition-laws-commence-soon is a brief summary of the changes to CCA from Harper


% UNUSED
% “When we look at retail banking markets in Australia we observe a number of indicators that, taken together, suggest that the current oligopoly structure is not vigorously competitive and has not been for some time,” the ACCC said.

% It said the big banks had maintained “significant market shares over a considerable time, largely unchallenged by smaller players, many of whom offer a less extensive range of products and services”.

% Moves to comprehensive credit reporting might help

% Possible options include open banking (more customer data)

%  (see PC submissions- eg APRA \url{http://www.apra.gov.au/Submissions/Documents/APRA-PC-Submission-FINAL-September2017.pdf})

% It also proposed reforms that would make it easier for customers to switch banks.

% “Although the large banks may resist such a measure, in the same way there was resistance to telephone number portability from Telstra 15 -20 years ago, any resistance should not be allowed to stifle an opportunity to empower new entrants and smaller players to more effectively challenge the large banks,” the regulator said.

% \url{http://thenewdaily.com.au/money/finance-news/2017/09/20/big-bank-oligopoly-slammed/}



%Many decisions are made not by the courts, not the regulator. 
%It is entirely possible that much of the clear and persistent market power of firms in some of the scale-economy sectors is fully consistent with competition law, even if it has not always been applied as forcefully as it could have been. The regulator should continue to XYZ. 

% Assuming that regulators do their part, what should policy makers do? They should seek to reduce entry barriers, including those imposed by regulation. 

% They should be alive to the possibility that some scale-economy sectors become tantamount to natural monopolies ..?


% sort of a good quote on `disruption'
% `"Are software companies going to figure out how to move into industries faster than the industries figure out how to be software companies?"
% (\url{http://www.afr.com/business/banking-and-finance/hedge-funds/future-fund-warns-rapid-technological-change-will-hit-returns-20171117-gznpdm})

% `Prudent policymakers must reinvent antitrust for the digital age. That means being more alert to the long-term consequences of large firms acquiring promising start-ups. It means making it easier for consumers to move their data from one company to another, and preventing tech firms from unfairly privileging their own services on platforms they control (an area where the commission, in its pursuit of Google, deserves credit). And it means making sure that people have a choice of ways of authenticating their identity online.'%
%     \footnote{\url{https://www.economist.com/news/leaders/21707210-rise-corporate-colossus-threatens-both-competition-and-legitimacy-business}.}

% PC recommendations on IP%
% \footcite{PC-IP-2016}
% (case for tax policy here? \eg{} rent taxes?)
