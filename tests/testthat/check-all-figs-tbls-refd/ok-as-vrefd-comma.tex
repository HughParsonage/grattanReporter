\documentclass{article}

\begin{document}
Over the past decade population has grown faster in the cities. Suburbs within 5\,km of the city centre have very high average growth rates (\Vref{fig:population-grew-faster}).%
\footnote{This chapter uses Census data for population numbers. The exception is population growth by distance to CBD (\Vref{fig:population-grew-faster,fig:population-grew-faster-melbourne}) for which we use ATO data on number of people filing a tax return as a proxy for population because Census data is not easy to summarise this way. Over 13 million people filed a tax return in 2015, about 55 per cent of the population. See \Vref{sec:app-a-incomes} for more detail. The main inaccuracy in using tax filers as a proxy for population is that it understates populations in areas with more children, such as new suburbs on capital city fringes, and in areas with many full Age Pensioners, such as coastal retiree towns.}
Population also grew particularly quickly in some postcodes on the fringes of capital cities (20\,km to 50\,km from the city centre), with housing estates being built on what was previously farmland.

\begin{figure}
\caption{Population grew faster in areas close to capital city centres\label{fig:population-grew-faster}}
\units{Average annual growth in number of taxable individuals, 2003-04 to 2014-15}
\includegraphics[page=8]{atlas/chart_pack.pdf} 
\noteswithsource{The growth rate is calculated as the CAGR in the number of individuals filing a tax return 2003-04 to 2014-15. A small number of outliers have been excluded from the chart to aid readability.}{\textcite[][Table~8]{ATO2017Taxationstatistics2014-15}; Grattan analysis.}
\end{figure}

Populations grew quickly in only a few regional areas away from the coast, such as Ballarat and Bendigo, which are both within a two hour drive of Melbourne. At a more micro level, many regional centres have grown at the expense of the smaller towns.

\begin{figure}
\caption{Population grew faster in areas close to Melbourne\label{fig:population-grew-faster-melbourne}}
\units{Average annual growth in number of taxable individuals, 2003-04 to 2014-15, Victoria}
\includegraphics[page=9]{atlas/chart_pack.pdf} 
\noteswithsource{The growth rate is calculated as the CAGR in the number of individuals filing a tax return 2003-04 to 2014-15. A small number of outliers have been excluded from the chart to aid readability.}{\textcite[][Table~8]{ATO2017Taxationstatistics2014-15}; Grattan analysis.}
\end{figure}
\end{document}
